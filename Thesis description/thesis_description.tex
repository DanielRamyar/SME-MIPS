\title{Thesis description}
\author{Carl-Johannes Johnsen (grc421)}
%% Skabelon til LiCS-afleveringer

%%%%%%%%%%%%%%%%%%%%%%%%%%%%%%%%%%%%%%%%%%%%%%%%%%%%%%%%%%%%%%%
%% Begynd preamble
%%%%%%%%%%%%%%%%%%%%%%%%%%%%%%%%%%%%%%%%%%%%%%%%%%%%%%%%%%%%%%%
\documentclass[a4paper]{article}

%% Til at tegne træer!
\usepackage{tikz}
%% Til at kunne have billeder
\usepackage{graphicx}
%% Til at kunne have source code
\usepackage{listings}
\lstset{
  breaklines=true,
  keepspaces=true,
  frame=ltrb,
  framesep=1pt,
  commentstyle=\color{Brown},
  basicstyle=\ttfamily\footnotesize,
  numbers=left,
  title=\lstname,
  columns=fullflexible,
  extendedchars=\true,
  inputencoding=ansinew,
}

%% Font and input encoding
%% Tillad æøå
\usepackage[T1]{fontenc}
\usepackage[utf8x]{inputenc}
%% Babel (language)
\usepackage[UKenglish]{babel} % If you write in English
\usepackage[UKenglish]{isodate}
%\usepackage{parskip}
\usepackage{booktabs}
%\usepackage[danish]{babel} % Hvis du skriver på dansk

%% Til links
\usepackage{hyperref}
\usepackage{subfig}

%% AMS-Math packages
\usepackage{amsmath}
\usepackage{amssymb}
\usepackage{amsthm}
\newtheorem{theorem}{Theorem}
%% Extra symbols that we almost always need
\usepackage{stmaryrd}
\usepackage{color}
\usepackage{url}
\usepackage{semantic}
\usepackage{fancyref}
\usepackage{enumitem}

%%%%%%%%%%%%%%%%%%%%%%%%%%%%%%
%% Skift sidenumrene ud med X/total (lettere at rette :-)
\usepackage{lastpage}
\makeatletter
\renewcommand{\@oddfoot}{\hfil \thepage{}/\pageref{LastPage} \hfil}
\renewcommand{\@evenfoot}{\hfil \thepage{}/\pageref{LastPage} \hfil}
\makeatother
%%%%%%%%%%%%%%%%%%%%%%%%%%%%%%

% Til klasse diagrammer
\usepackage{pgf-umlcd}
\renewcommand{\umldrawcolor}{black}
\renewcommand{\umlfillcolor}{white}
\let\classoperation\operation

% forkortelse af texttt!
\renewcommand{\bf}{\textbf}
\renewcommand{\tt}{\texttt}
\renewcommand{\sf}{\textsf}
\renewcommand{\it}{\textit}
\newcommand{\E}{\mathbb{E}}

% Forkortelser som bruger tt!
\newcommand{\tif}{\tt{ if }}
\newcommand{\tthen}{\tt{ then }}
\newcommand{\telse}{\tt{ else }}
\newcommand{\tfalse}{\tt{ false }}

%%%%%%%%%%%%%%%%%%%%%%%%%%%%%%%%%%%%%%%%%%%%%%%%%%%%%%%%%%%%%%%
%% Slut preamble -- herunder følger selve dokumentet!
%%%%%%%%%%%%%%%%%%%%%%%%%%%%%%%%%%%%%%%%%%%%%%%%%%%%%%%%%%%%%%%
\sloppy
\begin{document}
\maketitle

%The field 'Statement of dissertation' is limited to 2000 characters and allows
%no special characters and no formatting. Instead the student is encouraged to
%write a separate thesis description (see below) and write "See attached thesis
%description" into the field.
%Guidelines for the Thesis Description
%Normally a master thesis description is 1-3 pages and includes a
%background/motivation section (which also introduces key concepts and
%terminology), a concrete problem statement, a list of maybe 4-6
%project-specific learning objectives.  Optionally it can also include a list
%of expected major tasks to be performed (with time estimates) and a couple of
%central literature references. Figures, formulas, diagrams, etc. may be used
%to clarify the description. It's largely up to the supervisor to determine
%what makes the best sense for a particular thesis.

%Implementing a MIPS processor using SME
%Synchronous Message Exchange (SME) is a programming model, which is suitable
%for developing hardware models. This project aims to develop a MIPS processor
%using SME, documenting each step. Additionally, the material should be
%suitable for an introductory approach to generating actual processor hardware.

\section*{Background}
Synchronous Message Exchange (SME) \cite{sme} is a programming model, which is
similar to the Communicating Sequential Processes (CSP) model \cite{csp}. The
key differences are that SME is globally synchronous, has broadcasting channels
and a hidden clock. As such, SME is more suitable for devoloping hardware
models than CSP, and is also an easier approach for programmers to generating
hardware models.

MIPS (Microprocessor without Interlocked Pipeline Stages) is a reduced
instruction set computer (RISC) instruction set architecture (ISA) \cite{ark}.
The reason for selecting the MIPS instruction set, is due it being the
architecture taught in the Machine Architecture class (ARK) at DIKU.

As taught in ARK, the basic MIPS processor consists of 5 stages: Instruction
Fetch (IF), where the instruction is fetched from memory pointed to by an
instruction pointer. Instruction Decode (ID), where the instruction is decoded,
and where the register file is accessed. Execute (EX), where the actual
computation is made. Memory (MEM), where data is read or written to memory.
Finally, Write Back (WB), which takes the result, which is either from the EX
or the MEM stage, and writes it back to the register file, if needed.

In the basic processor, one instruction is executed per clock cycle, and as
such, the clock speed is determined by the amount of time needed to execute one
instruction. By introducing a pipeline, the clock speed can be increased, as
each instruction is divided into chunks.

By introducing a pipeline into the circuit, additional problems arise: Data
hazards, where instructions depend on the result of a previous instruction,
whose result is not ready in the registers. Branch prediction, since we need to
compute the branch condition before branching, branch prediction is needed in
order to reduce flushing the pipeline too much.

For making a hardware prototype, the circuit is written in VHDL (Very high
speed integrated circuit Hardware Description Language), which is then written
onto an FPGA (Field Programmable Gate Array).

\section*{Motivation and problem description}
The ARK class only looks at the theory of hardware, i.e. how the logic works,
how the different units are connected, and how the hardware affects the
software.  Therefore there is a gap in the education on the challenges of
generating actual hardware.

This project aims to develop teaching material on how to design actual
processor hardware by using SME. It will follow the same approach as the ARK
course, and should be suitable for people with a computer science background
such as myself.

\section*{Learning objectives}
By the end of this project I expect to have obtained the ability to:
\begin{enumerate}
    \item reason about differences in a software and a hardware program

    \item describe an integrated circuit using SME

    \item describe the evaluation of the correctness of an integrated circuit

    \item communicate hardware design to software people
\end{enumerate}

\section*{Major tasks to be performed along with time scheduling}
Each step in this schedule will follow the same procedure: Studying the needed
subject, implementing the feature/unit and finally documenting the
process/writing report. The time division of the procedure will be 10\%
reading, 45\% implementation and 45\% documentation.

My plan of subjects are as follows:
\begin{description}
    \item[Boolean circuits] - This is meant to be an introduction to SME and
        hardware design. I will try to implement a few logic gates, implement
        communication amongst these and finally validate the correctness of the
        implementation.

    \item[Full adder] - Now that we have some basic circuits, we can combine
        these to make a full adder. Although this is not needed with SME, it
        will give an insight on constructing arithmetic circuitry.

    \item[Single cycle MIPS CPU] - With the adder in place, I can begin
        constructing the basic CPU. As taught in ARK, I start by constructing
        the single cycle CPU, i.e. in one cycle exactly one instruction is
        executed. As such, I need to implement more of the processor units,
        e.g. instruction decoding, the register file, the ALU. When I have each
        step covered, connecting them into the basic CPU should be
        straightforward.

    \item[Pipelined MIPS CPU] - As mentioned in background, the single cycle
        CPU is not very efficient. So finally, I will introduce pipelines in
        the CPU, in order to increase the performance. Again, as mentioned, I
        will also need to handle the problems introduced by pipelining.
\end{description}
\includegraphics[width=\linewidth]{gantt2.png}

\begin{thebibliography}{9}
    \bibitem{sme} Brian Vinter \& Kenneth Skovhede. \it{Synchronous Message
        Exchange for Hardware Designs}. \copyright\ The authors and Open
        Channel Publishing Ltd. 2014.
    \bibitem{csp} C.A.R. Hoare. \it{Communicating Sequential Processes}.
        \copyright\ ACM 1978
    \bibitem{ark} David A. Patterson \& John L. Hennessy. \it{Computer
        Organization and Design - The Hardware/Software Interface (Revised 4th
        edition)}.  \copyright\ Elsevier 2012
\end{thebibliography}

\end{document}
