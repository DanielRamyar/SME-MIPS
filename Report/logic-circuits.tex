In this section, we will be looking at some basic combinatorial circuits. We
start by looking at some logic gates, which implement some basic boolean
functions. Then we will combine these basic gates into more complex networks: A
decoder, which expands an $n$-bit input into $2^n$ outputs. A half adder, which
takes two binary inputs and computes the sum and the carry of the two. A full
adder, which does the same operation as the half adder, but with an additional
third binary input. Finally, an $n$-bit adder, by combining a chain of a half
adder followed by $n-1$ full adders.

For each of these combinatorial circuits, we go through the theory behind it,
describe the procedure of translating the theory into an SME network and
finally how to test and verify the SME networks.

\subsection{Basic logic gates}
A logic gate is a circuit abstraction, which has inputs and outputs, and
computes the logic function that corresponds to the gates name, i.e. its output
values are based upon the input values. We are going to implement the following
logic gates:

\begin{description}
    \item[\texttt{AND}] - outputs \texttt{1} iff. all of its inputs are
        \texttt{1}, otherwise it outputs \texttt{0}.

    \item[\texttt{OR}] - outputs \texttt{1} if one or more of its inputs are
        \texttt{1}, otherwise it outputs \texttt{0}.

    \item[\texttt{NOT}] -outputs the inverse of its input, i.e. \texttt{1}
        becomes \texttt{0} and \texttt{0} becomes \texttt{1}.

    \item[\texttt{XOR}] - outputs \texttt{1} iff exactly one of its inputs are
        \texttt{1}, otherwise it outputs \texttt{0}
\end{description}
The full truth table for all of the four logic gates with 2-bit input (except
for \texttt{NOT}, which only uses 1 bit) can be seen in Table
\ref{tab:truth-table}.

\begin{table}
    \centering
    \begin{tabular}{cc|cccc}
        \toprule
        \texttt{Bit1} & \texttt{Bit2} & \texttt{AND} & \texttt{OR} &
        \texttt{NOT} & \texttt{XOR} \\
        \midrule
        \texttt{0} & \texttt{0} & \texttt{0} & \texttt{0} & \texttt{1} &
            \texttt{0} \\
        \texttt{0} & \texttt{1} & \texttt{0} & \texttt{1} & \texttt{1} &
            \texttt{1} \\
        \texttt{1} & \texttt{0} & \texttt{0} & \texttt{1} & \texttt{0} &
            \texttt{1} \\
        \texttt{1} & \texttt{1} & \texttt{1} & \texttt{1} & \texttt{0} &
            \texttt{0} \\
        \bottomrule
    \end{tabular}
    \caption{The truth table for the four basic logic gates. Note: \texttt{NOT}
    is only considering \texttt{Bit1}.}
    \label{tab:truth-table}
\end{table}

\subsubsection*{Implementation}
Implementing each of these four logic gates is straightforward: There is an
input bus with two 1-bit values, a process for each of the gates, and an output
bus with a 1-bit value for each of the logic gates. Each process takes the two
bits from the input bus, computes their respective logical function and sends
the result out on the output bus.
% (See Figure \ref{fig:logic-gate}).
%
%\begin{figure}
%    \centering
%    \begin{tikzpicture}
%        \coordinate(input);
%        \node[block, right of=input] (A) {Logic gate};
%        \path[->] (input) edge node [midway, above] {input} (A);
%        \node[right of=A] (output) {};
%        \path[->] (A) edge node [midway, above] {output} (output);
%    \end{tikzpicture}
%    \caption{The structure of a Logic gate process}
%    \label{fig:logic-gate}
%\end{figure}

\subsubsection*{Testing}
To test the four processes, we construct a process, which sets the bits on the
input bus to all of the values in the truth table, and checks whether or not
each process puts the expected value from the truth table on their output bus.
How the processes are connected can be seen in Figure \ref{fig:logic-test}.

\begin{figure}
    \centering
    \begin{tikzpicture}[node distance=1.5cm]
        \node[block] (and) {\texttt{AND}};
        \node[block, right of=and] (or) {\texttt{OR}};
        \node[block, right of=or] (not) {\texttt{NOT}};
        \node[block, right of=not] (xor) {\texttt{XOR}};
        \node[above of=and] (input) at ($(or)!0.5!(not)$) {};
        \node[block, above of=input] (tester) {Tester};

        \path[-] (tester) edge node[midway, right] {input} (input.center);
        \path[draw, ->] (input.center) -| (and.north);
        \path[draw, ->] (input.center) -| (or.north);
        \path[draw, ->] (input.center) -| (not.north);
        \path[draw, ->] (input.center) -| (xor.north);

        \node[below of=and] (output) at ($(or)!0.5!(not)$) {};
        \node[right of=xor] (a) {output};

        \path[draw, -] (and.south) |- (output.center);
        \path[draw, -] (or.south) |- (output.center);
        \path[draw, -] (not.south) |- (output.center);
        \path[draw, -] (xor.south) |- (output.center);

        \path[draw, -] (output.center) -| (a.west);
        \path[draw, ->] (a.west) |- (tester.east);
    \end{tikzpicture}
    \caption{The structure of the test of the logic gates}
    \label{fig:logic-test}
\end{figure}

\subsection{Decoder}
% TODO 0-indexed!
A decoder takes an $n$-bit input, and produces an $2^n$-bit output, where the
bit corresponding to the input numbers binary representation is set to
\texttt{1}. E.g. if the input value is the binary representation of the number
5, then the 5th output bit will be \texttt{1}, and the rest will be \texttt{0}.

A decoder can be made from a set of \texttt{NOT} and \texttt{AND} gates. We
need to have $n$ \texttt{NOT} gates, and $2^n$ \texttt{AND} gates. For each
input, we split it into two, and send the copy to a \texttt{NOT} gate. Then for
each output, we attach an \texttt{AND} gate, and give it inputs corresponding
to the binary representation of the number %TODO mangler ord.
E.g. if we get the number 5, the binary representation is 101, i.e. the 5th
\texttt{AND} gate gets input from Bit0, \texttt{NOT} Bit1 and Bit2. An example
of a 2-bit decoder can be seen in Figure \ref{fig:2-bit-decoder}.

\begin{figure}
    \centering
    \begin{tikzpicture}[node distance=1.5cm]
        \node[block] (and0) {\texttt{AND}};
        \node[block, below of=and0] (and1) {\texttt{AND}};
        \node[block, below of=and1] (and2) {\texttt{AND}};
        \node[block, below of=and2] (and3) {\texttt{AND}};

        \node[right of=and0] (output0) {output0};
        \node[right of=and1] (output1) {output1};
        \node[right of=and2] (output2) {output2};
        \node[right of=and3] (output3) {output3};

        \path[draw, ->] (and0) -- (output0);
        \path[draw, ->] (and1) -- (output1);
        \path[draw, ->] (and2) -- (output2);
        \path[draw, ->] (and3) -- (output3);

        \node[empty, left of=and0] (andinp0) {};
        \node[empty, left of=and1] (andinp1) {};
        \node[empty, left of=and2] (andinp2) {};
        \node[empty, left of=and3] (andinp3) {};

        \node[block, left of=andinp0] (not0) {\texttt{NOT}};
        \node[block, left of=andinp3] (not1) {\texttt{NOT}};

        \node[left of=not0] (input0) {input0};
        \node[left of=not1] (input1) {input1};

        \path[draw, ->] (input0) -- (not0);
        \path[draw, ->] (input1) -- (not1);

        \path[draw, thick, -] (not0) -| (andinp0.155);
        \path[draw, thick, ->] (andinp0.155) -- (and0.155);
        \path[draw, thick, -] (not1.east) -| (andinp0.south);
        \path[draw, thick, ->] (andinp0.south) |- (and0.200);

        \path[draw, ->] (input0) |- (and1.155);
        \path[draw, thick, -] (not1.east) -| (andinp1.340);
        \path[draw, thick, ->] (andinp1.340) -- (and1.200);

        \path[draw, thick, -] (not0.east) -| (andinp2.155);
        \path[draw, thick, ->] (andinp2.155) -- (and2.155);
        \path[draw, ->] (input1.north) |- (and2.200);

        \path[draw, -] (input0) |- (andinp1.295);
        \path[draw, ->] (andinp1.295) |- (and3.155);
        \path[draw, -] (input1) |- (andinp2.200);
        \path[draw, ->] (andinp2.200) |- (and3.200);

    \end{tikzpicture}
    \caption{An 2-bit decoder made by \texttt{AND} and \texttt{NOT} gates.}
    \label{fig:2-bit-decoder}
\end{figure}

\subsubsection*{Implementation}
To implement a 2-bit decoder, we just need to connect logic gate processes,
which we already have. How to connect a 2-bit decoder can be seen in Figure
\ref{fig:2-bit-decoder}.

However, making a scalable decoder is not trivial, as SME requires everything
to be known at compile time, and we cannot make a generic process, as these
depend on the names of the different busses. To solve this problem, we can use
C\# templates. For each input bit, we create an input bus. Then, for each input
bus, we create an \texttt{NOT} gate process, which takes the input bus
corresponding to its index. Then, we create $2^n$ output busses, and for each
of these, we connect an \texttt{AND} gate with its corresponding output bus.
Finally, for each of these \texttt{AND} gates, we connect the busses whose
logical \texttt{AND} will produce a \texttt{1}. E.g. for \texttt{output0}, we
connect all the busses from all of the \texttt{NOT} gates, for
\texttt{output1}, we connect all the \texttt{NOT} gates, except from the bus
from the first \texttt{NOT} gate, as this should be the first input bus.

\subsubsection*{Testing}
To test the 2-bit decoder, we follow the same procedure as before, with a
tester process, which sends input to the logic gates, and verifies the output
from the gates. This is also what we need to do with the $n$-bit decoder.
However as with the actual $n$-bit decoder, we are going to need C\# templates.
Each of the two tester processes should send all possible inputs as input.

\subsection{Adder}
As with the decoder, an adder can be constructed by a combination of
\texttt{AND}, \texttt{OR} and \texttt{XOR} gates. An $n$-bit adder is a chain
of two major components: an half adder and a full adder.

The half adder is the initial component in the chain. It takes two binary
inputs, and outputs the sum and the carry of the addition (See Figure
\ref{fig:half-adder}).

The rest of the $n$-bit adder consists of a chain of full adders, that take
three inputs, A, B, and the carry from the previous link in the chain, and
outputs the sum and the carry of the addition (See Figure
\ref{fig:full-adder}).

The combination of the components can be seen in Figure \ref{fig:n-bit-adder}.

\begin{figure}
    \centering
    \begin{tikzpicture}[node distance=1.5cm]
        \node[block] (xor) {\texttt{XOR}};
        \node[block, below of=xor] (and) {\texttt{AND}};

        \node[empty, left of=xor] (xorin) {};
        \node[empty, left of=and] (andin) {};

        \node[left of=xorin] (inputa) {InputA};
        \node[left of=andin] (inputb) {InputB};

        \node[empty, right of=xor] (sum) {Sum};
        \node[empty, right of=and] (carry) {Carry};

        \path[draw, -] (inputa) -| (xorin.155);
        \path[draw, ->] (xorin.155) -- (xor.155);
        \path[draw, ->] (xorin.west) |- (and.155);

        \path[draw, thick, -] (inputb.east) -| (xorin.335);
        \path[draw, thick, ->] (xorin.335) -- (xor.205);
        \path[draw, thick, ->] (andin.east) |- (and.205);

        \path[draw, ->] (xor) -- (sum);
        \path[draw, ->] (and) -- (carry);
    \end{tikzpicture}
    \caption{An half adder composed of \texttt{XOR} and \texttt{AND} gates}
    \label{fig:half-adder}
\end{figure}

\begin{figure}
    \centering
    \begin{tikzpicture}[node distance=1.5cm]
        \node[block] (or) {\texttt{OR}};
        \node[empty, left of=or] (orin) {};
        \node[block, left of=orin] (and1) {\texttt{AND}};
        \node[block, above of=and1] (and0) {\texttt{AND}};
        \node[empty, left of=and0] (and0in) {};
        \node[block, above of=and0] (xor1) {\texttt{XOR}};
        \node[empty, left of=xor1] (xor1in) {};
        \node[block, left of=xor1in] (xor0) {\texttt{XOR}};
        \node[empty, left of=xor0] (xor0in) {};
        \node[empty, left of=xor0in] (inputa) {InputA};
        \node[empty, below of=inputa] (inputb) {InputB};
        \node[empty, below of=inputb] (inputc) {InputC};
        \node[empty, right of=inputc] (inputcout) {};
        \node[empty, right of=xor1] (sum) {};
        \node[empty, right of=sum] (summ) {};
        \node[empty, right of=summ] (summm) {Sum};
        \node[empty, right of=or] (carry) {Carry};

        \path[draw, -] (inputa.east) -| (xor0in.155);
        \path[draw, ->] (xor0in.155) -- (xor0.155);
        \path[draw, -] (inputb.east) -| (xor0in.335);
        \path[draw, ->] (xor0in.335) -- (xor0.205);

        \path[draw, -] (inputa.east) -| (inputcout.205);
        \path[draw, ->] (inputcout.205) -- (and1.205);
        \path[draw, -] (inputb.east) -| (inputcout.25);
        \path[draw, ->] (inputcout.25) -- (and1.155);

        \path[draw, thick, -] (inputc.east) -| (and0in.335);
        \path[draw, thick, ->] (and0in.335) -- (and0.205);
        \path[draw, thick, ->] (and0in.335) |- (xor1.205);

        \path[draw, -] (xor0.east) -- (xor1in.west);
        \path[draw, ->] (xor1in.west) |- (xor1.155);
        \path[draw, ->] (xor1in.west) |- (and0.155);

        \path[draw, -] (and1.east) -| (orin.205);
        \path[draw, ->] (orin.205) -- (or.205);
        \path[draw, ->] (xor1) -- (summm);
        \path[draw, -] (and0.east) -| (orin.25);
        \path[draw, ->] (orin.25) -- (or.155);
        \path[draw, ->] (or) -- (carry);
    \end{tikzpicture}
    \caption{A full adder composed of \texttt{AND}, \texttt{OR} and
    \texttt{XOR} gates}
    \label{fig:full-adder}
\end{figure}

\begin{figure}
    \centering
    \begin{tikzpicture}[node distance=1.5cm]
        \node[block] (half) {Half adder};
        \node[block, below of=half] (full1) {Full adder 1};
        \node[empty, below of=full1] (dot) {...};
        \node[block, below of=dot] (fulln) {Full adder $n-1$};

        \node[empty, left of=half] (spacingl) {};
        \node[empty, right of=half] (spacingr) {};

        \node[empty, left of=spacingl] (in0) {Input 0};
        \node[empty, below of=in0] (in1) {Input 1};
        \node[empty, below of=in1] (vertspacel) {};
        \node[empty, below of=vertspacel] (inn) {Input $n$};

        \node[empty, right of=spacingr] (out0) {Sum 0};
        \node[empty, below of=out0] (out1) {Sum 1};
        \node[empty, below of=out1] (vertspacer) {};
        \node[empty, below of=vertspacer] (outn) {Sum $n-1$};
        \coordinate[below of=fulln] (carry);

        \path[draw, ->] (in0) -- (half);
        \path[draw, ->] (in1) -- (full1);
        \path[draw, ->] (inn) -- (fulln);

        \path[draw, ->] (half) -- (out0);
        \path[draw, ->] (full1) -- (out1);
        \path[draw, ->] (fulln) -- (outn);

        \path[draw, ->] (half) -- node [midway, right] {Carry 0} (full1);
        \path[draw, ->] (full1) -- node [midway, right] {Carry 1} (dot);
        \path[draw, ->] (dot) -- node [midway, right] {Carry $n-2$} (fulln);
        \path[draw, ->] (fulln) -- node [midway, right] {Carry $n-1$} (carry);
    \end{tikzpicture}
    \caption{An $n$-bit adder composed of a half adder, and $n-1$ full adders.
    Note: Input A and B are both inside the inputs for simplicity.}
    \label{fig:n-bit-adder}
\end{figure}

\subsubsection*{Implementation}
The half adder and the full adder is made as the decoder, in which we have the
basic logic gate processes, and connect them as specified in Figure
\ref{fig:half-adder} and Figure \ref{fig:full-adder}.

To make the $n$-bit adder, we use C\# templates like we did when constructing
the $n$-bit decoder. We program it so that each of the input bits has its own
bus, each of the output sums has its own bus, and each of the carrys have their
own bus. Then we start by making a half adder, which has the inputs: input A
bit 0 and input B bit 0. The initial half adder outputs its sum on sum 0, and
outputs the carry on carry 0. Then we construct $n-1$ full adders, where full
adder $i$ (1-indexed) has the inputs: input A bit $i$, input B bit $i$ and
carry $i-1$. Each full adder also has output $i$ and carry $i$ as output.

\subsubsection*{Testing}
To test the adder, we construct a tester process as before. We start by testing
the two subcomponents, the half and the full adder, by giving each every
possible input, and verifying the output. For the
$n$-bit adder, we make a function that takes an integer as input, and sends it
along the corresponding input wires. We also make a similar function that
takes the values from the output wires and packs it into an integer. There are
two tests: the simplest addition (2+2), overflow, and finally a series of
random numbers.
