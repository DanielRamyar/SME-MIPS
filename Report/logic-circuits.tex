In this section, I will be looking at some basic combinatorial circuits. I
start by looking at some logic gates, which implement some boolean functions.
% TODO: omskriv vil jeg mene!
All of the considered values in the system are binary, e.g. the logic gates
works on \texttt{1}s and \texttt{0}s.

\subsection{Basic logic gates}
Start by defining the basic logic gates, which are common for most circuits. A
logic gate is a circuit abstraction, which has inputs and outputs. Its output
values are based upon the input values.

\begin{description}
    \item[\texttt{AND}] - outputs \texttt{1} iff. all of its inputs are
        \texttt{1}, otherwise it outputs \texttt{0}.

    \item[\texttt{OR}] - outputs \texttt{1} if one or more of its inputs are
        \texttt{1}, otherwise it outputs \texttt{0}.

    \item[\texttt{NOT}] -outputs the inverse of its input, i.e. \texttt{1}
        becomes \texttt{0} and \texttt{0} becomes \texttt{1}.

    \item[\texttt{XOR}] - outputs \texttt{1} iff exactly one of its inputs are
        \texttt{1}, otherwise it outputs \texttt{0}
\end{description}
The full truth table for all of the four logic gates can be seen in Table
\ref{tab:truth-table}

\begin{table}
    \centering
    \begin{tabular}{cc|cccc}
        \toprule
        \texttt{Bit1} & \texttt{Bit2} & \texttt{AND} & \texttt{OR} &
        \texttt{NOT} & \texttt{XOR} \\
        \midrule
        false     & false     & false    & false   & true     & false \\
        false     & true      & false    & true    & true     & true  \\
        true      & false     & false    & true    & false    & true  \\
        true      & true      & true     & true    & false    & false \\
        \bottomrule
    \end{tabular}
    \caption{The truth table for the four basic logic gates. Note: \texttt{NOT}
    is only considering \texttt{Bit1}.}
    \label{tab:truth-table}
\end{table}

\subsubsection{Implementation}
Implementing each of these four logic gates is quite simple: There is an input
bus with two 1-bit values, a process for each of the gates, and an output bus
with a 1-bit value for each of the logic gates. % (See Figure
%\ref{fig:logic-gate}).
%
%\begin{figure}
%    \centering
%    \begin{tikzpicture}
%        \coordinate(input);
%        \node[block, right of=input] (A) {Logic gate};
%        \path[->] (input) edge node [midway, above] {input} (A);
%        \node[right of=A] (output) {};
%        \path[->] (A) edge node [midway, above] {output} (output);
%    \end{tikzpicture}
%    \caption{The structure of a Logic gate process}
%    \label{fig:logic-gate}
%\end{figure}

\subsubsection{Testing}
To test the four processes, I have made a process, which sets the bits to all
of the values in the truth table, and checks whether or not each process
outputs the expected value from the truth table. How the processes are
connected can be seen in Figure \ref{fig:logic-test}.

\begin{figure}
    \centering
    \begin{tikzpicture}[node distance=1.5cm]
        \node[block] (and) {\texttt{AND}};
        \node[block, right of=and] (or) {\texttt{OR}};
        \node[block, right of=or] (not) {\texttt{NOT}};
        \node[block, right of=not] (xor) {\texttt{XOR}};
        \node[above of=and] (input) at ($(or)!0.5!(not)$) {};
        \node[block, above of=input] (tester) {Tester};

        \path[-] (tester) edge node[midway, right] {input} (input.center);
        \path[draw, ->] (input.center) -| (and.north);
        \path[draw, ->] (input.center) -| (or.north);
        \path[draw, ->] (input.center) -| (not.north);
        \path[draw, ->] (input.center) -| (xor.north);

        \node[below of=and] (output) at ($(or)!0.5!(not)$) {};
        \node[right of=xor] (a) {output};

        \path[draw, -] (and.south) |- (output.center);
        \path[draw, -] (or.south) |- (output.center);
        \path[draw, -] (not.south) |- (output.center);
        \path[draw, -] (xor.south) |- (output.center);

        \path[draw, -] (output.center) -| (a.west);
        \path[draw, ->] (a.west) |- (tester.east);
    \end{tikzpicture}
    \caption{The structure of the test of the logic gates}
    \label{fig:logic-test}
\end{figure}

\subsection{Decoder}
% TODO 0-indexed!
The first combinatorial circuit I make is the decoder. An decoder takes an
$n$-bit input, and produces an $2^n$-bit output, where the bit corresponding to
the input numbers value is set to \texttt{1}. E.g. if the input value is the
binary representation of the number 5, then the 5th output bit will be
\texttt{1}, and the rest will be \texttt{0}.

A decoder can be made from a set of \texttt{NOT} and \texttt{AND} gates. We
need to have $n$ \texttt{NOT} gates, and $2^n$ \texttt{AND} gates. For each
input, we split it into two, and send the copy to a \texttt{NOT} gate. Then for
each output, we attach an \texttt{AND} gate, and give it inputs corresponding
to the binary representation of the number %TODO mangler ord.
E.g. if we get the number 5, the binary representation is 101, i.e. the 5th
\texttt{AND} gate gets input from Bit0, \texttt{NOT} Bit1 and Bit2. An example
of a 2-bit decoder can be seen in Figure \ref{fig:2-bit-decoder}.

\begin{figure}
    \centering
    \begin{tikzpicture}[node distance=1.5cm]
        \node[block] (and0) {\texttt{AND}};
        \node[block, below of=and0] (and1) {\texttt{AND}};
        \node[block, below of=and1] (and2) {\texttt{AND}};
        \node[block, below of=and2] (and3) {\texttt{AND}};

        \node[right of=and0] (output0) {output0};
        \node[right of=and1] (output1) {output1};
        \node[right of=and2] (output2) {output2};
        \node[right of=and3] (output3) {output3};

        \path[draw, ->] (and0) -- (output0);
        \path[draw, ->] (and1) -- (output1);
        \path[draw, ->] (and2) -- (output2);
        \path[draw, ->] (and3) -- (output3);

        \node[empty, left of=and0] (andinp0) {};
        \node[empty, left of=and1] (andinp1) {};
        \node[empty, left of=and2] (andinp2) {};
        \node[empty, left of=and3] (andinp3) {};

        \node[block, left of=andinp0] (not0) {\texttt{NOT}};
        \node[block, left of=andinp3] (not1) {\texttt{NOT}};

        \node[left of=not0] (input0) {input0};
        \node[left of=not1] (input1) {input1};

        \path[draw, ->] (input0) -- (not0);
        \path[draw, ->] (input1) -- (not1);

        \path[draw, thick, -] (not0) -| (andinp0.155);
        \path[draw, thick, ->] (andinp0.155) -- (and0.155);
        \path[draw, thick, -] (not1.east) -| (andinp0.south);
        \path[draw, thick, ->] (andinp0.south) |- (and0.200);

        \path[draw, ->] (input0) |- (and1.155);
        \path[draw, thick, -] (not1.east) -| (andinp1.340);
        \path[draw, thick, ->] (andinp1.340) -- (and1.200);

        \path[draw, thick, -] (not0.east) -| (andinp2.155);
        \path[draw, thick, ->] (andinp2.155) -- (and2.155);
        \path[draw, ->] (input1.north) |- (and2.200);

        \path[draw, -] (input0) |- (andinp1.295);
        \path[draw, ->] (andinp1.295) |- (and3.155);
        \path[draw, -] (input1) |- (andinp2.200);
        \path[draw, ->] (andinp2.200) |- (and3.200);

    \end{tikzpicture}
    \caption{An 2-bit decoder made by \texttt{AND} and \texttt{NOT} gates.}
    \label{fig:2-bit-decoder}
\end{figure}

%\subsubsection{$n$-bit decoder}
\subsubsection{Testing}
I have written both a 2-bit decoder, and an $n$-bit decoder. They are tested in
the same fashion as the logic gates, i.e. a tester process is connected to the
input and output bus. The test inputs every combination of numbers, i.e. $2^n$,
and checks after each input, that the correct output is set to \texttt{1} and
\texttt{0} respectively.

\subsection{Adder}
As with the decoder, an adder can be constructed by a combination of
\texttt{AND}, \texttt{OR} and \texttt{XOR} gates. An $n$-bit adder is a chain
of two major components: an half adder and a full adder.

The half adder is the initial component in the chain. It takes two binary
inputs, and outputs the sum and the carry of the addition (See Figure
\ref{fig:half-adder}.

The rest of the $n$-bit adder consists of a chain of full adders, that take
three inputs, A, B, and the carry from the previous link in the chain, and
outputs the sum and the carry of the addition (See Figure
\ref{fig:full-adder}).

The combination of the components can be seen in Figure \ref{fig:n-bit-adder}.


\begin{figure}
    \centering
    \begin{tikzpicture}[node distance=1.5cm]
        \node[block] (xor) {\texttt{XOR}};
        \node[block, below of=xor] (and) {\texttt{AND}};

        \node[empty, left of=xor] (xorin) {};
        \node[empty, left of=and] (andin) {};

        \node[left of=xorin] (inputa) {InputA};
        \node[left of=andin] (inputb) {InputB};

        \node[empty, right of=xor] (sum) {Sum};
        \node[empty, right of=and] (carry) {Carry};

        \path[draw, -] (inputa) -| (xorin.155);
        \path[draw, ->] (xorin.155) -- (xor.155);
        \path[draw, ->] (xorin.west) |- (and.155);

        \path[draw, thick, -] (inputb.east) -| (xorin.335);
        \path[draw, thick, ->] (xorin.335) -- (xor.205);
        \path[draw, thick, ->] (andin.east) |- (and.205);

        \path[draw, ->] (xor) -- (sum);
        \path[draw, ->] (and) -- (carry);
    \end{tikzpicture}
    \caption{An half adder composed of \texttt{XOR} and \texttt{AND} gates}
    \label{fig:half-adder}
\end{figure}

\begin{figure}
    \centering
    \begin{tikzpicture}[node distance=1.5cm]
        \node[block] (or) {\texttt{OR}};
        \node[empty, left of=or] (orin) {};
        \node[block, left of=orin] (and1) {\texttt{AND}};
        \node[block, above of=and1] (and0) {\texttt{AND}};
        \node[empty, left of=and0] (and0in) {};
        \node[block, above of=and0] (xor1) {\texttt{XOR}};
        \node[empty, left of=xor1] (xor1in) {};
        \node[block, left of=xor1in] (xor0) {\texttt{XOR}};
        \node[empty, left of=xor0] (xor0in) {};
        \node[empty, left of=xor0in] (inputa) {InputA};
        \node[empty, below of=inputa] (inputb) {InputB};
        \node[empty, below of=inputb] (inputc) {InputC};
        \node[empty, right of=inputc] (inputcout) {};
        \node[empty, right of=xor1] (sum) {};
        \node[empty, right of=sum] (summ) {};
        \node[empty, right of=summ] (summm) {Sum};
        \node[empty, right of=or] (carry) {Carry};

        \path[draw, -] (inputa.east) -| (xor0in.155);
        \path[draw, ->] (xor0in.155) -- (xor0.155);
        \path[draw, -] (inputb.east) -| (xor0in.335);
        \path[draw, ->] (xor0in.335) -- (xor0.205);

        \path[draw, -] (inputa.east) -| (inputcout.205);
        \path[draw, ->] (inputcout.205) -- (and1.205);
        \path[draw, -] (inputb.east) -| (inputcout.25);
        \path[draw, ->] (inputcout.25) -- (and1.155);

        \path[draw, thick, -] (inputc.east) -| (and0in.335);
        \path[draw, thick, ->] (and0in.335) -- (and0.205);
        \path[draw, thick, ->] (and0in.335) |- (xor1.205);

        \path[draw, -] (xor0.east) -- (xor1in.west);
        \path[draw, ->] (xor1in.west) |- (xor1.155);
        \path[draw, ->] (xor1in.west) |- (and0.155);

        \path[draw, -] (and1.east) -| (orin.205);
        \path[draw, ->] (orin.205) -- (or.205);
        \path[draw, ->] (xor1) -- (summm);
        \path[draw, -] (and0.east) -| (orin.25);
        \path[draw, ->] (orin.25) -- (or.155);
        \path[draw, ->] (or) -- (carry);
    \end{tikzpicture}
    \caption{A full adder composed of \texttt{AND}, \texttt{OR} and
    \texttt{XOR} gates}
    \label{fig:full-adder}
\end{figure}

\begin{figure}
    \centering
    \begin{tikzpicture}[node distance=1.5cm]
        \node[block] (half) {Half adder};
        \node[block, below of=half] (full1) {Full adder 1};
        \node[empty, below of=full1] (dot) {...};
        \node[block, below of=dot] (fulln) {Full adder $n-1$};

        \node[empty, left of=half] (spacingl) {};
        \node[empty, right of=half] (spacingr) {};

        \node[empty, left of=spacingl] (in0) {Input 0};
        \node[empty, below of=in0] (in1) {Input 1};
        \node[empty, below of=in1] (vertspacel) {};
        \node[empty, below of=vertspacel] (inn) {Input $n$};

        \node[empty, right of=spacingr] (out0) {Sum 0};
        \node[empty, below of=out0] (out1) {Sum 1};
        \node[empty, below of=out1] (vertspacer) {};
        \node[empty, below of=vertspacer] (outn) {Sum $n-1$};
        \coordinate[below of=fulln] (carry);

        \path[draw, ->] (in0) -- (half);
        \path[draw, ->] (in1) -- (full1);
        \path[draw, ->] (inn) -- (fulln);

        \path[draw, ->] (half) -- (out0);
        \path[draw, ->] (full1) -- (out1);
        \path[draw, ->] (fulln) -- (outn);

        \path[draw, ->] (half) -- node [midway, right] {Carry 0} (full1);
        \path[draw, ->] (full1) -- node [midway, right] {Carry 1} (dot);
        \path[draw, ->] (dot) -- node [midway, right] {Carry $n-2$} (fulln);
        \path[draw, ->] (fulln) -- node [midway, right] {Carry $n-1$} (carry);
    \end{tikzpicture}
    \caption{An $n$-bit adder composed of a half adder, and $n-1$ full adders.
    Note: Input A and B are both inside the inputs for simplicity.}
    \label{fig:n-bit-adder}
\end{figure}

\subsubsection{Testing}
I have tested both the half adder and the full adder, by giving each every
possible input, and for each, compared them to their expected output. For the
$n$-bit adder, I have made a function, that takes an integer as input, sends it
along the corresponding input wires. I have also made a similar function, that
takes the values from the output wires and packs it into an integer. I start by
testing if it can make one of the simplest additions: 2+2. Then I check if it
can overflow, i.e. set the Carry $n-1$ to \texttt{1}. Finally, I run a series
of random numbers through the adder, and check if the output is the expected
sum of the two numbers.
