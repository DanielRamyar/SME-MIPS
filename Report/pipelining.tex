In this section, I will be looking at pipelining the single cycle MIPS
processor, and handle the problems, which are introduced by pipelining. I
start by going through the background and motivation for pipelining, and then
proceed on how to extend the single cycle MIPS processor to have pipes.

As mentioned, pipelining introduces new problems to handle in the processor,
and I will solve it by adding two new components: the Forwarding Unit, which
forwards results from previous instructions to later instructions, and the
Hazard Detection Unit, which controls when to stall the pipeline. Throughout
each step, I will also be writing programs, in order to verify that the
processor behaves as specified. The source code for the pipelined processor
with forwarding and hazard detection is available at~\cite{ref:github} in
\texttt{sme/src/Examples/PipelinedMIPS/}.

\subsection{Introducing the pipes}\label{sec:intro-pipe}
I have the single cycle MIPS processor, which accepts some of the core integer
instruction set. However, it is not very efficient, as the clock rate of the
processor is determined by the longest possible path in the processor. A path
in a processor is the components that a signal goes through, until it reaches a
safe state. A safe state in a processor is a state, where the signals are safely
stored in either registers or memory. A longer path implies a lower clockrate.
So in order to increase the clock rate, I must decrease the longest path in the
processor. I solve this by introducing pipes.

Pipes are registers in the processor, where all the values computed so far are
temporarely stored. It takes all of its inputs, and holds them until the next
clock tick, where it will forward the values it is holding. This ensures that
the data does not have to travel as far, until it have reached a safe state.

In order to know where to put the pipes, I divide the processor into stages.  I
will use the same stages, as proposed in the architecture book~\cite{ref:ark},
i.e. divide the processor into 5 stages: Instruction Fetch (IF), Instruction
Decode (ID), Execution (EX), Memory (MEM) and Write Back (WB). A simplified
overview of the processor and its stages can be seen in Figure~\ref{fig:stages}.
In between each stage, I am going to insert a pipe, i.e. I am
going to insert 4 pipes. The simplified processor with pipes can be seen in
Figure~\ref{fig:pipes}.

Introducing pipes also introduces additional problems, which I will discuss
further in sections~\ref{sec:forw} and~\ref{sec:haz}

\begin{figure}
    \centering
    \scalebox{0.5}{
        \begin{tikzpicture}
            \node[block] (alu) at (0,0) {ALU};
            \node[block] (reg) at (-4,0) {Register File};
            \node[block] (memo) at (4,0) {Memory};
            \node[block] (jump) at (4,4.75) {Jump};
            \node[block] (imem) at (-8,0) {Instruction Memory};
            \node[block] (pc) at (-9, 2) {PC};
            \node[control] (cont) at (-4, 3) {Control};
            \node[mux] (wbm) at (8,1) {|};
            \node[mux] (addr) at (-4, 1) {|};

            \node[empty] (if) at (-8, 6) {\textbf{IF}};
            \node[empty] (id) at (-4, 6) {\textbf{ID}};
            \node[empty] (ex) at (0, 6) {\textbf{EX}};
            \node[empty] (mem) at (4, 6) {\textbf{MEM}};
            \node[empty] (wb) at (8, 6) {\textbf{WB}};

            \node[empty] (ifidt) at (-6,7) {};
            \node[empty] (idext) at (-2,7) {};
            \node[empty] (exmemt) at (2,7) {};
            \node[empty] (memwbt) at (6,7) {};
            \node[empty] (ifidb) at (-6,-3) {};
            \node[empty] (idexb) at (-2,-3) {};
            \node[empty] (exmemb) at (2,-3) {};
            \node[empty] (memwbb) at (6,-3) {};

            \path[draw, dashed, -] (ifidt) -- (ifidb);
            \path[draw, dashed, -] (idext) -- (idexb);
            \path[draw, dashed, -] (exmemt) -- (exmemb);
            \path[draw, dashed, -] (memwbt) -- (memwbb);

            \path[draw, ->] (-5.5,0) |- (addr.220);
            \path[draw, ->] (-5.5,0) |- (addr.140);
            \path[draw, ->] (pc) -| (imem);
            \path[draw, ->] (imem) -- (reg);
            \path[draw, ->] (-5.5,0) |- (cont);
            \path[draw, ->] (reg.15) -- (alu.151);
            \path[draw, ->] (reg.345) -- (alu.209);
            \path[draw, ->] (alu) -- (memo);
            \path[draw, ->] (-1,-0.31) -| (-1,-1) -- (2.5,-1) |- (memo.210);
            \path[draw, ->] (2.5, 0) |- (wbm.150);
            \path[draw, ->] (memo) -- (6.5, 0) |- (wbm.210);
            \path[draw, ->] (wbm) -| (9, -1.5) -| (reg.290);
            \path[draw, ->] (alu.30) -| (1,1) |- (4,2) -| (jump.290);
            \path[draw, ->] (jump) |- (0,5.5) -| (pc);
            \path[draw, ->] (-8,2) |- (jump.160);
            \path[draw, ->] (-5.5,3) |- (jump.200);
            \path[draw, ->] (addr) -| (-3, 1.5) -| (9.5,-2) -|
            (reg.270);

            %\path[draw, color=blue, ->] (cont) -- (reg);
            %\path[draw, color=blue, ->] (cont.10) -| (9.5,-2) -| (reg.250);
            \path[draw, color=blue, ->] (cont.10) -| (9.75,-2.25) -| (reg.250);
            \path[draw, color=blue, ->] (cont.310) -| (alu);
            \path[draw, color=blue, ->] (cont.330) -| (memo.115);
            \path[draw, color=blue, ->] (cont.350) -| (wbm);
            \path[draw, color=blue, ->] (cont.30) -| (jump.250);
            \path[draw, color=blue, ->] (cont) -- (addr);
        \end{tikzpicture}
    }
    \caption{The simplified single cycle processor, and its stages. The stage
    names are highlighted in \textbf{bold}. The stages are divided by dashed
    lines.}
    \label{fig:stages}
\end{figure}
\begin{figure}
    \centering
    \scalebox{0.5}{
        \begin{tikzpicture}
            \node[block] (alu) at (0,0) {ALU};
            \node[block] (reg) at (-4,0) {Register File};
            \node[block] (memo) at (4,0) {Memory};
            \node[block] (jump) at (4,4.75) {Jump};
            \node[block] (imem) at (-8,0) {Instruction Memory};
            \node[block] (pc) at (-9, 2) {PC};
            \node[control] (cont) at (-4, 3) {Control};
            \node[mux] (wbm) at (8,1) {|};
            \node[mux] (addr) at (-4, 1) {|};

            \node[empty] (if) at (-8, 6) {\textbf{IF}};
            \node[empty] (id) at (-4, 6) {\textbf{ID}};
            \node[empty] (ex) at (0, 6) {\textbf{EX}};
            \node[empty] (mem) at (4, 6) {\textbf{MEM}};
            \node[empty] (wb) at (8, 6) {\textbf{WB}};

            \path[draw, ->] (-5.5,0) |- (addr.220);
            \path[draw, ->] (-5.5,0) |- (addr.140);
            \path[draw, ->] (pc) -| (imem);
            \path[draw, ->] (imem) -- (reg);
            \path[draw, ->] (-5.5,0) |- (cont);
            \path[draw, ->] (reg.15) -- (alu.151);
            \path[draw, ->] (reg.345) -- (alu.209);
            \path[draw, ->] (alu) -- (memo);
            \path[draw, ->] (-1,-0.31) -| (-1,-1) -- (2.5,-1) |- (memo.210);
            \path[draw, ->] (2.5, 0) |- (wbm.150);
            \path[draw, ->] (memo) -- (6.5, 0) |- (wbm.210);
            \path[draw, ->] (wbm) -| (9, -1.5) -| (reg.290);
            \path[draw, ->] (alu.30) -| (1,1) |- (4,2) -| (jump.290);
            \path[draw, ->] (jump) |- (0,5.5) -| (pc);
            \path[draw, ->] (-8,2) |- (jump.160);
            \path[draw, ->] (-5.5,3) |- (jump.200);
            \path[draw, ->] (addr) -| (-3, 1.5) -| (9.5,-2) -|
            (reg.270);

            \path[draw, color=blue, ->] (cont.310) -| (alu);
            \path[draw, color=blue, ->] (cont.330) -| (memo.115);
            \path[draw, color=blue, ->] (cont.350) -| (wbm);
            %\path[draw, color=blue, ->] (cont.10) -| (9.5,-2) -| (reg.250);
            \path[draw, color=blue, ->] (cont.10) -| (9.75,-2.25) -| (reg.250);
            \path[draw, color=blue, ->] (cont.30) -| (jump.250);
            \path[draw, color=blue, ->] (cont) -- (addr);

            \node[block, minimum height=170, minimum width=10, fill=white] (ifid) at
            (-6,2.25) {};
            \node[block, minimum height=170, minimum width=10, fill=white] (idex) at
            (-2,2.25) {};
            \node[block, minimum height=180, minimum width=10, fill=white]
            (exmem) at (2,2) {};
            \node[block, minimum height=180, minimum width=10, fill=white]
            (exmem) at (6,2) {};
        \end{tikzpicture}
    }
    \caption{The simplified pipelined processor. The long bars in between each
    state are the pipes.}
    \label{fig:pipes}
\end{figure}

\subsubsection*{Implementation}
There are two ways to implement pipes in SME: by using clocked busses, or by
using clocked processes. If the bus only traverses two stages, then I can use
clocked busses, as the semantics of a clocked bus is exactly the same as a
pipe (i.e. the value is not passed along the bus, until the clock ticks).
However, if the bus traverses more than two stages, I am going to need
additional processes, as the clocked bus only stores its value for one clock
tick. Furthermore, if the only change to the bus is the given
\texttt{ClockedBus} attribute, then determining whether or not a bus is part of
a pipe can be problematic. As such, I have the pipe and its busses in their own
class, as it then receives its own namespace, and the code then becomes more
readable.

Let us take a subset of the IF/ID pipe, as an example. I assume that each
state is in its own classes, i.e. \texttt{IF} and \texttt{ID}, and that the
\texttt{IF} stage has the \texttt{Instruction} bus, which the \texttt{ID} stage
reads from. Then, by adding a subclass to the \texttt{IF} stage, the name of
the bus that the \texttt{ID} class calls is converteted from
\texttt{IF.Instruction} to \texttt{IF.Pipe.Instruction}, emphasizing that the
value is now fetched from the piped bus.

Introducing the pipes in this manner is fairly straightforward, as I just
add 4 classes, each with the same set of busses as the 'previous' stage outputs
to the 'next' stage, and each with a register process, which forwards all the
values from the 'previous' stages busses, onto its own busses with the same
name.

This process can be repeated for all of the required pipes. By following the
division of the components, as proposed in the architecture book~\cite{ref:ark},
there is only one really tricky part of pipelining: the Jump Unit. The
processor does not know whether to jump until the \texttt{MEM} stage, as the
computations are made in the \texttt{EX} stage, and the conditions are computed
in the \texttt{MEM} stage. However, the Program Counter has to increment,
regardless of whether or not it should jump. As such, I should divide the Jump
Unit into its subcomponents, and place some of the logic in the \texttt{IF},
some in the \texttt{EX} stage and finally some of it in the \texttt{MEM} stage.

For the \texttt{IF} stage, the incrementer and a multiplexor should be placed
inside the stage. The multiplexor should take an address computed from the
\texttt{MEM} stage, and the incremented Program Counter, and based on whether
or not the instruction that has reached the \texttt{MEM} stage was a branch or
jump instruction, it should choose the computed address.

The core computation of the jump and branch instructions should be placed in
the \texttt{EX} stage. As such, the \texttt{EX} stage should compute both the
branch address and the jump address.

Finally, the \texttt{MEM} stage should hold the decision logic. First, it
should determine if the instruction was an branch instruction, and whether or
not the condition has been satisfied, in which case, the address forwarded to
the \texttt{IF} stage should be the branch address. If this was not the case,
the computed jump address should be the one forwarded.

Finally, in the single cycle MIPS processor, I introduced the Write Buffer, in
order to remove the cycle from and to the Register File. However, by
introducing pipes, I have introduced a new buffer, and thus the Write Buffer
can be removed.

\subsubsection*{Testing}
To test the processor, I can use any of the programs, that I have previously
written. However, since I have pipelined the processor, I need to insert
bubbles, in order for data to be available for each instruction. A bubble is a
No Operation (\texttt{nop}) instruction, which performs no operation, and does
not modify neither the Register File nor the Memory.

I am going to implement a simple program, which is easy to verify: a small
loop, which computes $n$ fibonacci numbers, and places them in memory. As with
the simple cycle, I am going to give pseudo low level C code:
\begin{lstlisting}
void init(int *arr) {
    *(arr)   = 1;
    *(arr+1) = 1;
}

void loop(int *arr, int n) {
    int i, tmp1, tmp2, tmp3;
    for (i = 0; i < n; i++) {
        tmp1 = *(arr+i);
        tmp2 = *(arr+i+1);
        tmp3 = tmp1 + tmp2;
        *(arr+i+2) = tmp3;
    }
}
\end{lstlisting}
Note that for verification in C, the size of the array should be $n+2$, due to
initialization. Furthermore, when I port it to MIPS assembly, after each
instruction, I insert four \texttt{nop}'s, to ensure the data is ready for the
next instruction. When the program has run, the $n+2$ fibonacci numbers should
be in memory, at the given address.

\subsection{Forwarding}\label{sec:forw}
By introducing pipes, I also introduced data hazards and control hazards. I
will go through control hazards in Section~\ref{sec:haz}. Data hazards are when
one instruction writes to a register, that a following instruction reads from.
This is not a problem in the single cycle processor, as all data have been
written to registers in the same clock cycle. This is not the case in the
pipelined processor, e.g. the data might reside in the MEM stage, when it is
needed in the EX stage.

I can eliminate some of the data hazards by implementing an additional unit:
the Forwarding Unit. The rest of the data hazards will be handled by hazard
detection in Section~\ref{sec:haz}. The Forwarding Unit looks at the register
addresses used by the instruction in the EX stage, and checks if it corresponds
with the register write address of either the instruction in the MEM stage, or
the instruction in the WB stage. If they correspond, it forwards the value from
either the MEM or the WB stage to the EX stage. The overview of the simplified
processor with the forwarding unit can be seen in Figure~\ref{fig:forw}.

I could also have gone with the bubble approach, i.e. insert bubbles every
time there is a data hazard. However, this would not be optimal, as this would
result in several clock ticks, where the processor is idle.

\begin{figure}
    \centering
    \scalebox{0.5}{
        \begin{tikzpicture}
            \node[block] (alu) at (0.75,0) {ALU};
            \node[block] (reg) at (-4,0) {Register File};
            \node[block] (memo) at (4,0) {Memory};
            \node[block] (jump) at (4,4.75) {Jump};
            \node[block] (imem) at (-8,0) {Instruction Memory};
            \node[block] (pc) at (-9, 2) {PC};
            \node[control] (cont) at (-4, 3) {Control};
            \node[mux] (wbm) at (8,1) {|};
            \node[block] (forw) at (0, -3) {Forwarding Unit};
            \node[mux] (forwa) at (-0.25, 0.3) {|};
            \node[mux] (forwb) at (-0.75, -0.3) {|};
            \node[mux] (addr) at (-4, 1) {|};

            \node[empty] (if) at (-8, 6) {\textbf{IF}};
            \node[empty] (id) at (-4, 6) {\textbf{ID}};
            \node[empty] (ex) at (0, 6) {\textbf{EX}};
            \node[empty] (mem) at (4, 6) {\textbf{MEM}};
            \node[empty] (wb) at (8, 6) {\textbf{WB}};

            \path[draw, ->] (pc) -| (imem);
            \path[draw, ->] (imem) -- (reg);
            \path[draw, ->] (-5.5,0) |- (cont);
            \path[draw, ->] (-5.5,0) |- (addr.220);
            \path[draw, ->] (-5.5,0) |- (addr.140);
            %\path[draw, ->] (reg.15) -- (alu.151);
            \path[draw, ->] (reg.15) -- (forwa);
            \path[draw, ->] (forwa) --  (alu.151);
            \path[draw, ->] (reg.345) -- (forwb);
            \path[draw, ->] (forwb) -- (alu.209);
            \path[draw, ->] (alu) -- (memo);
            \path[draw, ->] (0,-0.3) |- (0,-1) -- (2.75,-1) |- (memo.210);
            \path[draw, ->] (2.5, 0) |- (wbm.150);
            \path[draw, ->] (memo) -- (6.5, 0) |- (wbm.210);
            \path[draw, ->] (wbm) -| (9.25, -1.75) -| (reg.290);
            \path[draw, color=blue, dashed, ->] (alu.30) -| (1.5,1) |- (4,2) -| (jump.290);
            \path[draw, ->] (jump) |- (0,5.5) -| (pc);
            \path[draw, ->] (-8,2) |- (jump.160);
            \path[draw, ->] (-5.5,3) |- (jump.200);
            \path[draw, thick, ->] (-1.25,-1.75) |- (forwb.220);
            \path[draw, thick, ->] (-1.25,-1.75) |- (forwa.220);
            \path[draw, dashed, ->] (2.5, 0) |- (-1.5, -1.5) |- (forwb.140);
            \path[draw, dashed, ->] (-1.5, -1.5) |- (forwa.140);
            \path[draw, ->] (addr) -| (-3, 1.5) -| (9.5,-2) -|
            (reg.270);
            \path[draw, thick, ->] (9.5, -2) |- (forw.353);
            \path[draw, dashed, ->] (5,1.5) |- (forw.15);
            \path[draw, dashed, ->] (-5.5,0) |- (-1.75,-1) |- (forw.170);
            \path[draw, dashed, ->] (-5.5,0) |- (-2,-1.25) |- (forw.190);

            \path[draw, color=blue, ->] (cont.310) -| (alu);
            \path[draw, color=blue, ->] (cont.330) -| (memo.115);
            \path[draw, color=blue, ->] (cont.350) -| (wbm);
            \path[draw, color=blue, ->] (cont.10) -| (9.75,-2.25) -| (reg.250);
            \path[draw, color=blue, ->] (cont.30) -| (jump.250);
            \path[draw, color=blue, ->] (cont) -- (addr);
            \path[draw, color=blue, thick, ->] (9.75,-2.25) |- (forw.345);
            \path[draw, color=blue, dashed, ->] (5.25, 3.2) |- (forw.7);

            \path[draw, color=blue, dashed, ->] (forw.115) -- (forwa);
            \path[draw, color=blue, dashed, ->] (forw.144) -- (forwb);


            \node[block, minimum height=170, minimum width=10, fill=white] (ifid) at
            (-6,2.25) {};
            \node[block, minimum height=190, minimum width=10, fill=white] (idex) at
            (-2.5,1.75) {};
            \node[block, minimum height=180, minimum width=10, fill=white]
            (exmem) at (2,2) {};
            \node[block, minimum height=180, minimum width=10, fill=white]
            (exmem) at (6,2) {};
        \end{tikzpicture}
    }
    \caption{The simplified pipelined processor with forwarding.}
    \label{fig:forw}
\end{figure}

\subsubsection*{Implementation}
Implementing the Forwarding Unit should be done with an SME process. As it
interferes with the EX stage, it should be put in the EX stage. The unit should
be controlling two multiplexors. The first multiplexor should control whether
Output A should come from the ID pipe, the write data from the MEM stage or the
write data from the WB stage. The other multiplexor is analogous, but with
Output B. The two multiplexors should be controlled by the Forwarding Unit,
i.e. it should generate the control signals for the multiplexors based on the
two register read addresses of the current instruction in the EX stage, the
register write address and write enabled signal both from the MEM and the WB
stages.

\subsubsection*{Testing}
Testing is straightforward: I can just remove all of the \texttt{nop}'s from
the fibonacci program, except for those which come after either a load, a
branch or a jump, as these hazards are not handled yet.

\subsection{Hazard Detection}\label{sec:haz}
As mentioned in Section~\ref{sec:forw}, I cannot avoid all data hazards with
forwarding. This is because it only forwards to the EX stage. However,
if the instruction prior to the instruction in EX is a load, then the data wont
be available, until the load instruction has reached the WB stage. To handle
this, the processor needs to detect the hazard and insert a bubble, as this
will delay the instruction, so that when it has reached the EX stage, the load
instruction will be in the WB stage, and thus it can be forwarded. When
inserting a bubble in the middle of the pipeline, some of the pipes and
registers must also be stalled. Stalling is the action of outputting what is stored
in the register, but not updating it, i.e. outputting the same data in the
next clock tick.

I also need to be able to handle an additional type of hazard: control
hazards. Control hazards are when either jump or branch instructions are
executed. The problem is that the branch or jump is not performed until the MEM
stage, which means that the pipeline following the jump or branch instruction
may have been filled up by instructions, which should not be executed. To solve
this, it should detect the hazard, and in such case flush the pipeline.
Flushing the pipeline, is the action of resetting the registers in the pipes,
so that they output a \texttt{nop} instruction. The overview of the simplified
processor with the Hazard Detection Unit can be seen in Figure~\ref{fig:hazard}.

\begin{figure}
    \centering
    \scalebox{0.5}{
        \begin{tikzpicture}
            \node[block] (alu) at (0.75,0) {ALU};
            \node[block] (reg) at (-4,0) {Register File};
            \node[block] (memo) at (4,0) {Memory};
            \node[block] (jump) at (4,4.75) {Jump};
            \node[block] (imem) at (-8,0) {Instruction Memory};
            \node[block] (pc) at (-9, 2) {PC};
            \node[control] (cont) at (-4, 3) {Control};
            \node[mux] (wbm) at (8,1) {|};
            \node[block] (forw) at (0, -3) {Forwarding Unit};
            \node[mux] (forwa) at (-0.25, 0.3) {|};
            \node[mux] (forwb) at (-0.75, -0.3) {|};
            \node[mux] (addr) at (-4, 1) {|};
            \node[block] (haz) at (-4,6.5) {Hazard Detectection};

            \node[empty] (if) at (-8, 8) {\textbf{IF}};
            \node[empty] (id) at (-4, 8) {\textbf{ID}};
            \node[empty] (ex) at (0, 8) {\textbf{EX}};
            \node[empty] (mem) at (4, 8) {\textbf{MEM}};
            \node[empty] (wb) at (8, 8) {\textbf{WB}};

            \path[draw, ->] (pc) -| (imem);
            \path[draw, ->] (imem) -- (reg);
            \path[draw, ->] (-5.5,0) |- (cont);
            \path[draw, ->] (-5.5,0) |- (addr.220);
            \path[draw, ->] (-5.5,0) |- (addr.140);
            %\path[draw, ->] (reg.15) -- (alu.151);
            \path[draw, ->] (reg.15) -- (forwa);
            \path[draw, ->] (forwa) --  (alu.151);
            \path[draw, ->] (reg.345) -- (forwb);
            \path[draw, ->] (forwb) -- (alu.209);
            \path[draw, ->] (alu) -- (memo);
            \path[draw, ->] (0,-0.3) |- (0,-1) -- (2.75,-1) |- (memo.210);
            \path[draw, ->] (2.5, 0) |- (wbm.150);
            \path[draw, ->] (memo) -- (6.5, 0) |- (wbm.210);
            \path[draw, ->] (wbm) -| (9.25, -1.75) -| (reg.290);
            \path[draw, color=blue, dashed, ->] (alu.30) -| (1.5,1) |- (4,2) -| (jump.290);
            \path[draw, ->] (jump) |- (0,5.5) -| (pc);
            \path[draw, ->] (-8,2) |- (jump.160);
            \path[draw, ->] (-5.5,3) |- (jump.200);
            \path[draw, thick, ->] (-1.25,-1.75) |- (forwb.220);
            \path[draw, thick, ->] (-1.25,-1.75) |- (forwa.220);
            \path[draw, dashed, ->] (2.5, 0) |- (-1.5, -1.5) |- (forwb.140);
            \path[draw, dashed, ->] (-1.5, -1.5) |- (forwa.140);
            \path[draw, ->] (addr) -| (-3, 1.5) -| (9.5,-2) -|
            (reg.270);
            \path[draw, thick, ->] (9.5, -2) |- (forw.353);
            \path[draw, dashed, ->] (5,1.5) |- (forw.15);
            \path[draw, dashed, ->] (-5.5,0) |- (-1.75,-1) |- (forw.170);
            \path[draw, dashed, ->] (-5.5,0) |- (-2,-1.25) |- (forw.190);

            \path[draw, color=blue, ->] (cont.310) -| (alu);
            \path[draw, color=blue, ->] (cont.330) -| (memo.115);
            \path[draw, color=blue, ->] (cont.350) -| (wbm);
            \path[draw, color=blue, ->] (cont.10) -| (9.75,-2.25) -| (reg.250);
            \path[draw, color=blue, ->] (cont.30) -| (jump.250);
            \path[draw, color=blue, ->] (cont) -- (addr);
            \path[draw, color=blue, thick, ->] (9.75,-2.25) |- (forw.345);
            \path[draw, color=blue, dashed, ->] (5.25, 3.2) |- (forw.7);

            \path[draw, color=blue, dashed, ->] (forw.115) -- (forwa);
            \path[draw, color=blue, dashed, ->] (forw.144) -- (forwb);

            % Hazard
            \path[draw, thick, ->] (-0.75, 1.5) |- (haz); % addr
            \path[draw, thick, color=blue, ->] (-1, 2.37) |- (haz.350); % mem
            \path[draw, thick, ->] (-0.5, 5.5) |- (haz.10); % pcsrc
            \path[draw, thick, ->] (-5.5, 4.55) -- (haz.200);% reada readb
            \path[draw, dashed, color=blue, -] (haz) -- (-4, 5.75);

            \node[block, minimum height=170, minimum width=10, fill=white] (ifid) at
            (-6,2.25) {};
            \node[block, minimum height=190, minimum width=10, fill=white] (idex) at
            (-2.5,1.75) {};
            \node[block, minimum height=180, minimum width=10, fill=white]
            (exmem) at (2,2) {};
            \node[block, minimum height=180, minimum width=10, fill=white]
            (memwb) at (6,2) {};

            \path[draw, dashed, color=blue, ->] (-4, 5.75) -| (exmem);
            \path[draw, dashed, color=blue, ->] (-4, 5.75) -| (idex);
            \path[draw, dashed, color=blue, ->] (-4, 5.75) -| (ifid.92);
            \path[draw, dashed, color=blue, ->] (-4, 5.75) -| (pc.115);
            \path[draw, color=blue, ->] (haz) -| (ifid.88);
            \path[draw, color=blue, ->] (haz) -| (pc.65);
        \end{tikzpicture}
    }
    \caption{The simplified pipelined processor with forwarding and hazard
    detection.}
    \label{fig:hazard}
\end{figure}

\subsubsection*{Implementation}
The hazard detection should be implemented in its own SME process: the Hazard
Detection Unit.

To solve the data hazards, it needs to read the \texttt{memread} control flag
from the EX stage, the register write address from the EX stage and the two
source register addresses from the ID stage. If the \texttt{memread} flag has
been set, and either of the source registers match the destination register,
then the ID/EX pipe should be flushed, the IF/ID pipe should be stalled, and
the PC register should be stalled.

To solve the control hazards, it should read the \texttt{PCSrc} flag from the
MEM stage (the one used by the PC register, to multiplex between the
incremented address, and the jump/branch address), and if it is set, then the
IF/ID, ID/EX and EX/MEM pipes should be flushed. Note: the PC register should
read normally in the case of a flush, even though it might have to stall, as
the stall is detected in the pipe that is being flushed.

\subsubsection*{Testing}
Testing the new hazard detection is straightforward, as the processor should
now be able to handle all of the previous programs, albeit in a larger amount
of clock ticks, compared to the single cycle processor. The performance of the
processor with hazard detection can be seen in Table~\ref{tab:perf-pipe}.

\begin{table}
    \centering
    \begin{adjustbox}{center}
    \begin{tabular}{rlllllllllll}
        & & \texttt{MARS} & & & \texttt{SME} \\
        \hline
        & & \# CT & time (ms) & CR (hz) &
            \# CT & time (ms) & CR (hz) \\
        \hline
        Towers of Hanoi & $n=5$ &
            719        & 585        & $\sim$1229 &
            720 - 1058 & 516 - 1190 & $\sim$1395 - $\sim$889 \\

        Quicksort & $n=8$ &
            483       & 852       & $\sim$829 &
            484 - 763 & 375 - 895 & $\sim$1290 - $\sim$852 \\

        Fib no optimization & $n=10$ &
            220       & 584       & $\sim$376 &
            221 - 251 & 191 - 356 & $\sim$1157 - $\sim$753 \\

        Fib forward & $n=10$ &
            98        & 586       & $\sim$167 &
            100 - 130 & 119 - 209 & $\sim$840 - $\sim$ 622 \\

        Fib hazard & $n=10$ &
            84       & 588       & $\sim$142 &
            86 - 126 & 113 - 212 & $\sim$761 - $\sim$594 \\
        \hline
    \end{tabular}
    \end{adjustbox}
    \caption{Performance of the Quicksort, Towers of Hanoi and different
    versions of the fibonacci programs. CT is clock ticks. CR is clockrate.
    In the columns with multiple values, the left value is the performance of
    the single cycle processor and the right is the performance of the pipelined
    processor. The benchmark was performed on a laptop with an Intel Core
    i5-5300U (2.3 GHz).}
    \label{tab:perf-pipe}
\end{table}


By looking at the performance in Table~\ref{tab:perf-pipe}, I see that the
performance of the pipelined processor is worse both in execution time, clock
ticks and clockrate. The additional clock ticks are expected of the pipelined
processor, as there is an wind-up time for filling the pipeline, and as I loose
some instructions by stalling and flushing.

By looking at the fibonacci program with hazard detection, I see that the
single cycle processor uses 86 clock ticks. It runs the loop 10 times and for
each branch it flushes the pipeline, i.e. losing $10 \times 3 = 30$
instructions. Furthermore, there is a dependency following a load instruction,
producing a stall in each loop, i.e. losing an additional 10 instructions. By
summing this up, I get $86 + 30 + 10 = 126$, which is exactly the
amount of instructions performed by the pipelined processor.

The execution time of the SME simulation, and as such also the simulation
clockrate, is worse. This is due to the SME simulation having more work to do,
as I have increased the amount of processes and busses. There should not be any
decrease in performance when synthesized to FPGA, but rather an increase.
As mentioned in Section~\ref{sec:intro-pipe}, it is because I should be able
to increase the clockrate, compared to the single cycle implementation.
