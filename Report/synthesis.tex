In this section, I will be describing the steps required to implement an SME
network onto an actual FPGA. I start with the initial logic gates design, as it
is very simple to verify, and because it does not have any requirements
regarding clocking.

Then I will be implementing the single cycle MIPS processor, and finally show
how to communicate with the hardware implemented on the FPGA. This section will
be very hardware specific, as I have not had time to verify or develop an
approach, which will run on additional hardware.

\subsection{General workflow}
TODO figur! SME > ghdl > vivado func > vivado post-impl > bitstream > interface func >
export > SDK.

In order to implement hardware onto an FPGA, one must describe the hardware
using a hardware description language such as VHDL or Verilog. As mentioned
before, SME can be transpiled into VHDL, and as such provides a high level
approach for hardware design. Therefore, the first thing to do is to write an
SME network, verify that it runs as expected, and then checking that the
transpiler does not fail to transpile.

Once the VHDL has been generated, the VHDL can be verified by using
\texttt{ghdl}. \texttt{ghdl} is an commandline simulator for VHDL. SME provides
a testbench and a \texttt{Makefile} for running the testbench using
\texttt{ghdl}. As such, to verify the generated VHDL, one only needs to run
\texttt{make} inside the output \texttt{vhdl/} folder.

Processor is clocked at 5mhz!
