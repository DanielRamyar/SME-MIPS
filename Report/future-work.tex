The initial step would be to add an AXI interface to the Pipelined processor.
The approach is the same as with the single cycle processor. It would be
interesting to see whether or not the increased clock is applicable to the
additional AXI interface.
% The initial step would be to synthesize and implement the pipelined processor.
% Given that I have already stated the approach for the single cycle processor,
% this should not be too hard. It could be interesting to see, if it can be
% clocked higher and in that case how much. Furthermore, it would also be more
% usable, as it should have access to more memory.

Inspired by the manycore architecture of the SW26010 processors in the
TaihuLight supercomputer~\cite{ref:supercomputer}, I want to see how many
cores I am able to fit onto a single FPGA. By doing this, I also want to
introduce scratchpad memory, instead of focusing on the traditional memory
hierarchy, as this is what is being used in the SW26010.

It would also be interesting to extend the processor core, to a superscalar
processor, i.e. by introducing multiple execution paths. This could both be by
introducing additional ALUs, e.g. floating point, or by introducing
coprocessors, which are specialized for certain tasks, such as the SME network
of a student project, which solved partial differential equations and fast
fourier transforms.

A final suggestion would be to extend the processor with the required
components, such that a minimal operating system could be compiled and run on
the processor. This would be useful in specialized hardware, if a device should
run purely in its own environment.
