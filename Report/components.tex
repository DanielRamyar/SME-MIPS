In this section, we will be looking at the major components of the MIPS
processor. We look at the components before wiring them together into a
processor, as it keeps each step more isolated and simple.

For each component, we will go through the theory of the component, then we
will look at translating theory into an SME implementation, and finally the
verification of the implementations. When we are verifying the implementation,
we should construct a tester process, just as with the basic logic circuits.
The order in which the components are mentioned is derived from chapter 4.3 in
the book\cite{ref:ark}, and should be the order of implementation.

By the end of this section, we should have all of the required resources for
building our single cycle MIPS processor using SME.

\subsection{Instruction Memory}
The first component that we need is a memory unit to hold the instructions of a
program, and to supply instructions at a given address. We will also need a
register holding the current address in the program called the Program Counter
(PC). The memory unit has one input: the address from the PC, and one output:
the read instruction. The PC has one input: the input address. The Instruction
Memory, PC and their connections can be seen in Figure \ref{fig:inst}.
\begin{figure}
    \centering
    \begin{tikzpicture}
        \node[block] (inmem) at (0,0) {Instruction Memory};
        \node[block] (pc) at (-2, 2) {PC};
        \node[empty] (prog) at (1.5,2) {Program Counter};
        \node[empty] (inst) at (4,0) {Instruction};
        \node[empty] (addr) at (-4,2) {Address};

        \path[draw, ->] (pc) -| (inmem);
        \path[draw, ->] (inmem) -- (inst);
        \path[draw, ->] (addr) -- (pc);
    \end{tikzpicture}
    \caption{The Instruction Memory}
    \label{fig:inst}
\end{figure}

\subsubsection*{Implementation \& Testing}
Both the PC and the Instruction Memory should be SME processes. We could have
combined the two components into a single SME process, but we are trying to
keep each process as simple as possible. The input bus for the PC, the Program
Counter bus and the instruction bus, should all contain an \texttt{uint} value.

The PC should have a single \texttt{uint} value holding the current address,
and on each tick, it should output its stored value, and store the new input
value. It should also be clocked, as this is the starting point of every
instruction.

The Instruction Memory should have an \texttt{byte} array, as this will make
indexing into the array simpler. Usually, the Instruction Memory and the Memory
Unit share the same address space. However, for simplicity, we give each
component its own memory. On each tick, the Instruction Memory should take the
input address, read the byte at the given index and the following 3 bytes, and
finally pack all 4 bytes together into an \texttt{uint}, and place it on the
instruction bus.

We could have had the memory as an \texttt{uint} array, but then we would have
to divide the address by 4, since the addresses in the MIPS processor are byte
addressed. We could modify the processor to deal with addresses in an indexing
manner, however this would make programs harder to translate, as the compiled
programs can have absolute addresses.

Testing the Instruction Memory is very trivial. The program should be hardcoded
into the \texttt{byte} array in the memory. The tester process should then send
values to the PC, and verify that the output on the instruction bus, matches
the value at the given address.

\subsection{Register file}
The MIPS processor has 32 general-purpose 32-bit registers, which is stored in
a structure called the Register File. Each of the registers can be read from or
written to, except for register 0, which is immutable and always 0. It is the
first step in a memory hierarchy, and is thus the fastest memory available.
The registers are divided into groups based on their usage, but this does not
matter from a hardware perspective.

The Register File has 5 inputs: Read Address A, Read Address B, Write Enabled
(\texttt{RegWrite}), Write Address and Write Data. The Register File has two
outputs: Output A and Output B.  There are two stages of the Register File:
reading and writing. In the reading stage, it takes the value in the register
at the address of Read Address A, and outputs it on Output A, and vice versa
for Read Address B and Output B. In the writing stage, if the Write Enabled
flag is set, it takes the value from the Write Data bus, and stores it in the
register with the address given in the Write Address bus.

We need to be careful of the order in which we read and write from the Register
File. We need to make sure that when an instruction reads from the Register
File, it always gets the latest data, i.e. if an instruction reads from the
same register as a previous instruction writes to, it should get the newly
written value. This is easy to fix in the single cycle processor, as we just
need to write before reading. We cannot do it in the reverse order, as the
Register File might then output old values. The Register File and its inputs
and outputs can be seen in Figure \ref{fig:register}.

\begin{figure}
    \centering
    \begin{tikzpicture}[node distance=2cm]
        \node[empty] (inputa) {Input A};
        \node[empty, below of=inputa] (inputb) {InputB};

        \node[empty, right of=inputa] (spacing) at ($(inputa)!0.5!(inputb)$) {};
        \node[block, right of=spacing] (register) {Register};
        \node[empty, below of=register] (writedata) {Write data};
        \node[empty, left of=writedata] (write) {Write register};
        \node[empty, right of=writedata] (writeenabled) {Write enabled};

        \node[empty, right of=inputa] (space) {};
        \node[empty, right of=space] (spacee) {};
        \node[empty, right of=spacee] (spaceee) {};
        \node[empty, right of=spaceee] (outputa) {Output A};
        \node[empty, below of=outputa] (outputb) {Output B};
        \node[empty, left of=outputb] (bspace) {};

        \path[draw, -] (inputa) -| (spacing.north);
        \path[draw, ->] (spacing.north) |- (register.155);
        \path[draw, -] (inputb) -| (spacing.south);
        \path[draw, ->] (spacing.south) |- (register.205);

        \path[draw, color=blue, -] (write) |- (writedata.135);
        \path[draw, color=blue, ->] (writedata.135) -- (register.225);
        \path[draw, color=blue, ->] (writedata) -- (register);
        \path[draw, color=blue, -] (writeenabled) |- (writedata.45);
        \path[draw, color=blue, ->] (writedata.45) -- (register.315);

        \path[draw, -] (register.335) -| (bspace.north);
        \path[draw, ->] (bspace.north) |- (outputb);
        \path[draw, -] (register.25) -| (spaceee.south);
        \path[draw, ->] (spaceee.south) |- (outputa);
    \end{tikzpicture}
    \caption{The register file}
    \label{fig:register}
\end{figure}

\subsubsection*{Implementation \& Testing}
The Register File should be an SME process. The collection of registers should
be an \texttt{uint} array of length 32. The address busses should all contain
an \texttt{byte} value, as the number of addressable registers never exceed
$2^8=256$. The output busses and the Write Data bus should all contain an
\texttt{uint} value. Finally, the \texttt{RegWrite} bus should contain a
\texttt{bool}.

On each clock tick, the register file should check if the
\texttt{RegWrite} flag is set, in which case it should take the value from the
Write Data bus, and store in the register at the address from the Write Address
bus. Then it should take the value in the register at the address from the Read
Address A bus, and output onto the Output A bus, and analougously for the Read
Address B bus and the Output B bus.

It should be noted that if the write address is 0, then the write should be
ignored, as register 0 is immutable.

Testing the register file is very trivial. We start by sending some values
on the write data bus, along with some addresses and the \texttt{RegWrite}
flag set.

Then we just try to send some addresses on the Read address A and Read
address B buses, and verify that the register file outputs the values
stored at these addresses. It is also important to verify the behaviour of
register zero.

\subsection{ALU}
\label{sec:alu}
The ALU (Arithmetic Logic Unit) is the part of the processor, which makes the
actual computation. It takes three inputs: InputA, InputB and an ALU Opcode
indicating which computation to perform. It has two outputs: The result of the
computation, and a zero flag indicating whether or not the result of the
computation was 0. The overview of the component and its inputs and outputs can
be seen in Figure \ref{fig:alu}.

The ALU starts by looking at the value in the ALU Opcode, as this determines
which operation to perform. Then it reads the values from Input A and Input B,
and performs the operation specified by the ALU Opcode. Finally, it outputs the
result, and a flag indicating whether the result was 0.

\begin{figure}
    \centering
    \begin{tikzpicture} [node distance=1.5cm]
        \node[empty] (ina) {Input A};
        \node[empty, below of=ina] (inb) {Input B};
        \node[empty, right of=inputa] (spacing) at ($(ina)!0.5!(inb)$) {};
        \node[block, right of=spacing] (alu) {ALU};
        \node[empty, right of=ina] (align0) {};
        \node[empty, right of=align0] (align1) {};
        \node[empty, right of=align1] (align2) {};
        \node[empty, right of=align2] (result) {Zero};
        \node[empty, below of=result] (zero) {Result};
        \node[empty, left of=zero] (align3) {};
        \node[empty, below of=alu] (aluop) {ALUOp};

        \path[draw, -] (ina.east) -| (spacing.155);
        \path[draw, ->] (spacing.155) -- (alu.155);
        \path[draw, -] (inb.east) -| (spacing.205);
        \path[draw, ->] (spacing.205) -- (alu.205);
        \path[draw, color=blue, ->] (aluop) -- (alu);
        \path[draw, color=blue, -] (alu.25) -| (align2.east);
        \path[draw, color=blue, ->] (align2.east) -- (result);
        \path[draw, -] (alu.335) -| (align3.east);
        \path[draw, ->] (align3.east) -- (zero);
    \end{tikzpicture}
    \caption{The ALU}
    \label{fig:alu}
\end{figure}

\subsubsection*{Implementation \& Testing}
To implement the ALU, we start by making an \tt{enum}, so that the code becomes
more human readable. Each entry in the \tt{enum} corresponds to a computation
that the ALU should perform. We could have followed the ALU Opcode given in the
book\cite{ref:ark}, however it states that its encoding of the ALU Opcode is
generated using a CAD tool, and as such we will generate it ourselves too.

We start by implementing the same instructions as the book\cite{ref:ark}
proposes in chapter 4.1: \texttt{add}, \texttt{sub}, \texttt{and}, \texttt{or},
\texttt{slt}, \texttt{sw}, \texttt{lw} and \texttt{beq}. To perform these
instructions, the ALU should be able to perform addition, subtraction, logical
\tt{AND}, logical \tt{OR} and the comparison less than.

The two input busses and the Result bus should all contain an \texttt{uint}
value. The ALU Opcode bus should contain a \texttt{byte} value. Finally, the
Zero bus should contain a \texttt{bool} value.

Constructing the ALU process in SME is straightforward, it reads the ALU Opcode
from the ALUOp bus, and then it performs a \texttt{switch} on the ALU Opcode.
For the instructions that it accepts, it takes the input from the two input
busses, do the computation, and output the result on the Result bus. SME will
handle the the actual computation, and as such we do not need to do anything
but do the right computation in C\#. Note: it is important to cast the
\texttt{uint} input to \texttt{int}, if the operation is signed.

Finally, the ALU should set the flag on the Zero bus, depending on whether or
not the result of the computation was 0.  How the ALU opcode is encoded is
described in Section \ref{sec:alu-control}.

There are no special cases to consider when testing the ALU, we should just
test whether or not it can compute the right result, based on the inputs.

\subsection{Sign Extend}
The Sign Extend is used for extracting the 16-bit values from the instruction,
called the Immediate. It has one input: the lower 16 bits from the
instruction, and one output: the 32-bit sign extended value. It takes its
input, which is 16-bit, and converts it into a 32-bit value, extending the sign
if present. The Sign Extend can be seen in Figure \ref{fig:sign}

\begin{figure}
    \centering
    \begin{tikzpicture}
        \node[block] (sign) at (0,0) {Sign Extend};
        \node[empty] (imm) at (-3,0) {Immediate};
        \node[empty] (immout) at (4,0) {Sign extended immediate};

        \path[draw, ->] (imm) -- (sign);
        \path[draw, ->] (sign) -- (immout);
    \end{tikzpicture}
    \caption{The Sign Extend}
    \label{fig:sign}
\end{figure}

\subsubsection*{Implementation \& Testing}
To implement the Sign Extend, we construct an SME process. The input bus should
contain a \texttt{short} value, and the output bus should contain an
\texttt{uint} value. On each clock tick, the SME proces takes the 16-bit
immediate, and outputs it on its 32-bit output bus. We could manually do the
sign extending, but it is simple to let C\# handle the sign extension for us.

Testing the Sign Extend is straightforward: send values on the input bus, and
verify that the same number is on the output bus.

\subsection{Memory unit}
The Memory Unit is the main memory of the processor. It is the second step of
the memory hierarchy, and is thus slower than the Register File, but can
contain a lot more data. The processor can either read or write to the Memory
Unit. The addresses for the memory are byte addresses and word aligned, i.e. in
the 32-bit processor, the word size is 32 bit, and therefore the address should
be dividable by 4.

The Memory Unit has four inputs: Address, Data, \texttt{MemRead} and
\texttt{MemWrite}. It has a single output: Read Data. In one clock cycle, the
Memory Unit either reads or writes. The Memory Unit can be seen in Figure
\ref{fig:mem}.
\begin{figure}
    \centering
    \begin{tikzpicture}
        \node[block] (mem) at (0,0) {Memory};
        \node[empty] (addr) at (-3,0.4) {Address};
        \node[empty] (data) at (-3,-0.4) {Data};
        \node[empty] (memread) at (-1.5, 1) {\texttt{MemRead}};
        \node[empty] (memwrite) at (1.5, 1) {\texttt{MemWrite}};
        \node[empty] (readdata) at (3, 0) {Read Data};

        \path[draw, ->] (addr) -- (mem.154);
        \path[draw, ->] (data) -- (mem.206);
        \path[draw, color=blue, ->] (memread) -| (mem.110);
        \path[draw, color=blue, ->] (memwrite) -| (mem.70);
        \path[draw, ->] (mem) -- (readdata);
    \end{tikzpicture}
    \caption{The Memory Unit}
    \label{fig:mem}
\end{figure}

\subsubsection*{Implementation \& Testing}
To implement the Memory Unit, we construct an SME process. The Address, Data
and Output busses should all contain an \texttt{uint} value. The two control
busses \texttt{MemRead} and \texttt{MemWrite} should both contain a
\texttt{bool} value.

As with the Instruction Memory, we are going to need a chunk of memory. Like
the Instruction Memory, the memory chunk should be a \texttt{byte} array, and
should read and write in the same manner, i.e. either packing or unpacking 4
bytes, from and to memory.

On each clock tick, the process should check if the \texttt{MemRead} flag is
set, in which case it should read the value on the Address bus, and output the
value stored in memory at the read address.

If that was not the case, it should check if the \texttt{MemWrite} flag is set,
in which case it should read the value at the Address bus and the Data bus,
store the read data in memory at the read address.

To test the Memory Unit, we should just try to store some values, and check if
we can fetch the same values again.

\subsection{Splitter}
The book\cite{ref:ark} takes the instruction read from the Instruction Memory,
and splits it out to the different components. One way of handling this could
be to send the instruction bus to all the components that needs something from
the instruction. However, then the resulting VHDL might send all 32 bits to
multiple components, when 5 bits might have been sufficient. As such, we
construct our own component for splitting up the instruction.

The splitter is a very simple component: It takes the instruction, which has
been fetched from memory, and divides it into chunks for the different parts of
the decoding. The instruction is partitioned as followed (the bits are
inclusive):
\begin{itemize}
    \item Opcode - bits 26-31
    \item Read Address A - bits 21-25
    \item Read Address B - bits 16-20
    \item Write Address - bits 11-15
    \item Immediate - bits 0-15
    \item Funct - bits 0-5
\end{itemize}
The Splitter can be seen in Figure \ref{fig:split}
\begin{figure}
    \centering
    \begin{tikzpicture}
        \node[block] (split) at (0,0) {Splitter};
        \node[empty] (inst) at (-3,0) {Instruction};
        \node[empty] (op) at (4,1.25) {Opcode};
        \node[empty] (ra) at (4,0.75) {Read Address A};
        \node[empty] (rb) at (4,0.25) {Read Address B};
        \node[empty] (wa) at (4,-0.25) {Write Address};
        \node[empty] (imm) at (4,-0.75) {Immediate};
        \node[empty] (fun) at (4,-1.25) {Funct};

        \path[draw, ->] (inst) -- (split);
        \path[draw, ->] (split.25) -| (1.5, 1) |- (op);
        \path[draw, ->] (split.15) -| (1.75,0.5) |- (ra);
        \path[draw, ->] (split.5) -| (2,0.125) |- (rb);
        \path[draw, ->] (split.355) -| (2,-0.125) |- (wa);
        \path[draw, ->] (split.345) -| (1.75,-0.5) |- (imm);
        \path[draw, ->] (split.335) -| (1.5, -1) |- (fun);
    \end{tikzpicture}
    \caption{The Splitter}
    \label{fig:split}
\end{figure}

\subsubsection*{Implementation \& Testing}
The implementation is straightforward: We construct an SME process, which takes
the instruction coming from the Instruction Memory, and extract the bits at the
indices, by using C\# bit hacking. The Input bus should contain an
\texttt{uint} value. The Immediate bus should contain an \texttt{short} value.
The rest of the busses should contain an \texttt{byte} value.  Finally, the
process should output the extracted values on the respective output busses.

Testing the Splitter requires no special procedures.

\subsection{Jump Unit}
The book\cite{ref:ark} describes how to implement \texttt{beq}, by the use of
multiple components. However, we should combine these components into a single
unit, the Jump Unit, to simplify the processor.

The Jump Unit is the one controlling which instruction to load next. It takes
four inputs: sign extend, Zero, \texttt{beq} and the PC. It produces one
output: the new PC.
\begin{figure}
    \centering
    \begin{tikzpicture}
        \node[block] (and) at (0,0) {\texttt{AND}};
        \node[block] (inc) at (-2,1) {+4};
        \node[block] (add) at (2,1) {+};
        \node[empty] (pc) at (-4,1) {PC};
        \node[empty] (beq) at (-0.5,-1.5) {\texttt{beq}};
        \node[empty] (zero) at (0.5,-1.5) {Zero};
        \node[empty] (imm) at (4,1) {Immediate};
        \node[empty] (out) at (-4,2) {New PC};
        \node[mux] (mux) at (0,2) {|};

        \path[draw, ->, color=blue] (beq.65) -- (and.245);
        \path[draw, ->, color=blue] (zero.115) -- (and.295);
        \path[draw, ->, color=blue] (and) -- (mux);
        \path[draw, ->] (pc) -- (inc);
        \path[draw, ->] (inc) -- (add);
        \path[draw, ->] (imm) -- (add);
        \path[draw, ->] (add.north) |- (mux.45);
        \path[draw, ->] (1, 1) |- (mux.315);
        \path[draw, ->] (mux) -- (out);

        \draw (-2.8,-0.8) rectangle (2.8,2.8);
    \end{tikzpicture}
    \caption{The Jump Unit}
    \label{fig:jump}
\end{figure}

For the simple single cycle MIPS processor, it should only have support for
normal program traversal (i.e. execute the instructions in order) and the
\texttt{beq} instruction. We will be adding support for more branch and jump
instructions later.

For the simple traversal, the Jump Unit takes the previous PC, and increments
it by 4. For the \texttt{beq} instruction, it takes the value from the Sign
Extend, shifts it left by 2, and add that value to the incremented PC. Finally,
it chooses between the incremented PC and the added sign extend, based on the
\texttt{beq} and Zero flags. The Jump Unit and its internals can be seen in
Figure \ref{fig:jump}.

\subsubsection*{Implementation \& Testing}
We have the all of the logic described in the background. To implement the Jump
Unit, we implement all of the logic components as SME processes. We are going
to need 4 components: an +4 incrementer, an adder, an \texttt{AND} gate and a
multiplexor.

The Incrementer takes the PC bus as input, and produces a single output on the
inced bus. The two busses should both contain an \texttt{uint} value. On each
clock tick, it should take the value from the PC bus, add 4 to it and put it on
the inc bus.

The Adder takes the inc bus and the immediate bus from the Sign Extend as
input. It has a single output bus: the added bus. All of the busses should
contain an \texttt{uint} value. On each clock tick, the Adder should add its
two inputs together, and put the result on the add bus.

The \texttt{AND} gate is like in the logic gates example, but with the
\texttt{beq} and Zero as inputs, and its output should be on a new bus: anded.
All of its busses should contain a \texttt{bool} value.

Finally, we have the multiplexor, which takes the inced, added and anded busses
as input, and produces a single output: the new PC. The new PC bus should
contain an \texttt{uint} value. On each clock tick, if the flag from the anded
bus is set to \texttt{1}, then it should put the value from the added bus onto
the new PC bus. Otherwise it should output the value from the inced bus.

We could have made all the logic in C\#, however this makes it easier to
extend, as each subcomponent is very simple.

We test the Jump Unit by initially verifying that the PC increments as it
should, if the two control signals are \texttt{0}. Then it should be tested, if
the correct branching addresses are constructed.

\subsection{ALU control}\label{sec:alu-control}
The ALU control is used for generating the ALU Operation control code, which
the ALU uses for selecting which operation to perform. It takes two inputs: the
\texttt{ALUOp} code from the control unit, and the \texttt{funct} code from the
instruction. It produces a single output: The ALU Opcode. The ALU Control can
be seen in Figure \ref{fig:alu-cont}.
\begin{figure}
    \centering
    \begin{tikzpicture}
        \node[block] (alucont) at (0,0) {ALU Control};
        \node[empty] (aluop) at (-3, -1) {ALUOp};
        \node[empty] (funct) at (-3, -1.5) {Funct};
        \node[empty] (aluope) at (0, 2) {ALU Operation};

        \path[draw, ->, color=blue] (aluop) -| (alucont.250);
        \path[draw, ->] (funct) -| (alucont.290);
        \path[draw, ->, color=blue] (alucont) -- (aluope);
    \end{tikzpicture}
    \caption{The ALU Control}
    \label{fig:alu-cont}
\end{figure}

If the \texttt{ALUOp} indicates that the instruction is an R format
instruction, it uses the \texttt{funct} code for selecting the operation.
Otherwise it bases its output on the \texttt{ALUOp} code. We will return to
this component, when we need to extend the instruction set.

\subsubsection*{Implementation \& Testing}
The book\cite{ref:ark} describes the logic to implement the ALU Control.
However, like the ALU, it has been generated, and is difficult to extend. As
such, we construct our own ALU Control, and let the VHDL generator handle the
logic generation.

To make the source code more human readable, we should have two additional
\texttt{enum}s: one for the \texttt{ALUOp} code and one for the \texttt{funct}.
Then we construct an SME process, which has the connections as specified in the
background section. All of its busses should contain \texttt{byte} values.

On each clock tick, the ALU Control checks if the \texttt{ALUOp} code indicate
R format instruction or not. If the instruction is an R format, the process
should \texttt{switch} on the \texttt{funct} code. If not, it should
\texttt{switch} on the \texttt{ALUOp} code. In all cases in both
\texttt{switch}s, the ALU Control should output the ALU Operation corresponding
to the computation that the instruction expects.

To test the ALU Control, the tester process should try all possible
combinations, as there should not be too many combinations.

\subsection{Control Unit}
\label{sec:control-unit}
The control unit is part of the decoding step. It takes the opcode of the
instruction, and based on the opcode, it sets control flags used throughout the
processor. All of the control flags mentioned throughout the other components
are set by the Control Unit. It sets the following control flags:
\begin{description}
    \item[RegDst] Controls which part of the instruction that indicates which
        B register to read from.

    \item[Branch] Controls whether or not the instruction is a branch
        instruction.

    \item[MemRead] Controls whether or not there should be read from memory.

    \item[MemtoReg] Controls whether or not the value from memory should be
        stored in the register file.

    \item[ALUOp] Opcode indicating which operation should be performed in
        the ALU. It is send to the ALU Control for further processing.

    \item[MemWrite] Controls whether or not data should be written to memory.

    \item[ALUSrc] Controls whether the B input for the ALU should be the
        value read from the register file, or if it should be the value
        extracted from the instruction.

    \item[RegWrite] Controls whether or not data should be written to the
        register file.
\end{description}
Each of the control flags goes to their respective part of the processor.
We will return to this component, when we have to extend it to handle more
instructions. The control unit and its connections can be seen in Figure
\ref{fig:cont-unit}.
\begin{figure}
    \centering
    \begin{tikzpicture}
        \node[control] (cont)   at (0,  0)   {Control Unit};
        \node[empty] (opcode)   at (-3, 0)   {Opcode};
        \node[empty] (branch)   at (4,  1.75) {\texttt{Branch}};
        \node[empty] (regwrite) at (4,  1.25) {\texttt{RegWrite}};
        \node[empty] (memtoreg) at (4,  0.75) {\texttt{MemToReg}};
        \node[empty] (memwrite) at (4,  0.25) {\texttt{MemWrite}};
        \node[empty] (memread)  at (4, -0.25) {\texttt{MemRead}};
        \node[empty] (alusrc)   at (4, -0.75) {\texttt{ALUSrc}};
        \node[empty] (aluop)    at (4, -1.25) {\texttt{ALUOp}};
        \node[empty] (regdst)   at (4, -1.75) {\texttt{RegDst}};

        \path[draw, ->] (opcode) -- (cont);

        \path[draw, ->, color=blue] (cont.35)  -| (2,1.5) |- (branch);
        \path[draw, ->, color=blue] (cont.25)  -| (2.25,1) |- (regwrite);
        \path[draw, ->, color=blue] (cont.15)  -| (2.5,0.5) |- (memtoreg);
        \path[draw, ->, color=blue] (cont.5)   -| (2.75,0.125) |- (memwrite);
        \path[draw, ->, color=blue] (cont.355) -| (2.75,-0.125) |- (memread);
        \path[draw, ->, color=blue] (cont.345) -| (2.5,-0.5) |- (alusrc);
        \path[draw, ->, color=blue] (cont.335) -| (2.25,-1) |- (aluop);
        \path[draw, ->, color=blue] (cont.325) -| (2,-1.5) |- (regdst);
    \end{tikzpicture}
    \caption{The Control Unit}
    \label{fig:cont-unit}
\end{figure}

\subsubsection*{Implementation \& Testing}
As with the ALU Control, the book\cite{ref:ark} describes the logic needed to
implement this unit. However, again it is not trivial to extend later on, so we
construct our own Control Unit, and let the VHDL generator handle the logic
generation.

We construct an SME process, which has all of the busses described in the
background section. The input opcode bus and the ALUOp code bus should both
contain an \texttt{byte} value. The rest of the busses should all contain an
\texttt{bool} value.

As with the ALU Control, we should have an \texttt{enum} on the opcode, to make
it more human readable. We do not need one for the ALUOp, as the ALU Control
already described it.

On each clock tick, the Control Unit reads the opcode from the input bus. Then
it should \texttt{switch} on the read opcode, and set all of the flags
accordingly. How the flags should be set, depends on the instruction, and
should be straightforward, but time demanding.

As with the ALU Control, testing the Control Unit should be done by trying
every opcode available.

\subsection{Write back}
The final stage of the processor is the Write Back. Here, the values are sent
to the Register File for storing.

\subsubsection*{Implementation \& Testing}
Usually in the single cycle MIPS processor, there is nothing special in the
Write Back stage. However, in SME we are not allowed to have unclocked cycles,
and there is a cycle from the Register File, through the ALU and the Memory
Unit, and back to the Register File.

To solve this, we introduce a Write Buffer. The write buffer should be clocked,
and takes the Write Data, Write Register and \texttt{WriteEnabled} as input,
and produces the same output. On each clock tick, it should output its stored
values, and store its input values.

We could also have made one of the previous components clocked, however we do
not want to this, as this specific problem will be solved when we pipeline the
processor in section \ref{sec:pipelining}, and then the Write Buffer can be
removed again.

The Write Buffer and its connections can be seen in Figure \ref{fig:wb}.
\begin{figure}
    \centering
    \begin{tikzpicture}
        \node[block] (wb) at (0,0) {Write Buffer};
        \node[empty] (wr) at (0,2) {Write Register};
        \node[empty] (wd) at (3,0.5) {Write Data};
        \node[empty] (rw) at (3,-0.5) {\texttt{RegWrite}};
        \node[empty] (wr2) at (-3, 0.5) {Write Register};
        \node[empty] (wd2) at (-3, 0) {Write Data};
        \node[empty] (rw2) at (-3, -0.5) {\texttt{RegWrite}};

        \path[draw, ->] (wr) -- (wb);
        \path[draw, ->] (wd) -| (1.35, 0.45) |- (wb.20);
        \path[draw, color=blue, ->] (rw) -| (1.35, -0.45) |- (wb.340);
        \path[draw, ->] (wb) -- (wd2);
        \path[draw, ->] (wb.160) -| (-1.35,0.45) |- (wr2);
        \path[draw, color=blue, ->] (wb.200) -| (-1.35,-0.45) |- (rw2);
    \end{tikzpicture}
    \caption{The Write Buffer}
    \label{fig:wb}
\end{figure}
Testing the Write Buffer is very trivial: verify that it outputs its input from
the previous clock tick.
